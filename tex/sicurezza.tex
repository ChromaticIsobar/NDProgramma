% !TeX root = ../main.tex
\chapter{Dalmine +Sicura}
\section{Sicurezza}
Il tema della sicurezza richiede una visione interdisciplinare, capace di abbracciare diverse dimensioni ed avvalersi della collaborazione di tutti gli attori coinvolti ma anche della cittadinanza. Il nostro obiettivo è quello di introdurre misure strutturali che consentano di ridurre i rischi, piuttosto che intervenire in riparazione di eventi già avvenuti. Pertanto ci prefiggiamo di:

\fakeparagraph{Riqualificare aree urbane degradate}, attraverso il potenziamento dell'illuminazione pubblica e la manutenzione degli spazi pubblici, anche interloquendo con i soggetti privati proprietari di aree a rischio ecologico (e.g. area ex Cavalleri);

\fakeparagraph{Ostacolare l'abbandono dei rifiuti}, gli atti vandalici ed i danneggiamenti, incrementando la videosorveglianza comunale nelle zone sensibili; 

\fakeparagraph{Monitorare costantemente la regolarità e la trasparenza di ogni attività diretta al pubblico} (e.g. commerciale, culturale, industriale) che si tenga sul territorio comunale;

\fakeparagraph{Aumentare la sicurezza sulle strade}, migliorando la segnaletica e l'illuminazione in corrispondenza di attraversamenti pedonali e ciclabili, e prevedendo zone a traffico limitato, a velocità ridotta o chiuse al traffico in punti nevralgici e in orari critici; 

\fakeparagraph{Portare avanti progetti di educazione alla legalità} incentrati sulla cura del bene pubblico con il contributo della Polizia Locale, della Protezione Civile e di altri enti del territorio;

\fakeparagraph{Contrastare la ludopatia e le truffe} nei confronti degli anziani e delle persone fragili, grazie alla collaborazione con le Forze dell'Ordine;

\fakeparagraph{Incrementare il personale della Polizia Locale}, per garantire una maggiore e più capillare presenza sul territorio, anche in orari notturni e nel fine settimana, favorendo a tal fine la collaborazione con altri enti pubblici ed istituti privati;

\fakeparagraph{Aumentare la partecipazione, specialmente tra i giovani, alla Protezione Civile}, attraverso ulteriori bandi per il servizio civile ed incontri per sensibilizzare la cittadinanza sulle attività e le funzioni da essa svolte;

\fakeparagraph{Migliorare la formazione, le dotazioni ed i mezzi a disposizione delle Forze dell'Ordine e della Protezione Civile}. 
