% !TeX root = ../../main.tex
\section{Commercio}
Nell'ultimo decennio il forte sviluppo di attività commerciali di grande e media distribuzione a Dalmine ed in particolar modo verso l'asse della ex SS525 ha messo a dura prova il commercio locale, con la desertificazione dei centri storici e delle frazioni. In un simile contesto, il commercio di vicinato necessita di un intervento tempestivo e mirato, soprattutto considerato le dirette ricadute positive che quest'ultimo ha sul territorio, ovvero sull'economia, sull'occupazione e a livello sociale, in particolar modo in relazione agli anziani e alle fasce più deboli della popolazione.
Per questo motivo abbiamo identificato alcuni punti cardine su cui lavorare che sono riassunti qui di seguito:

\fakeparagraph{Prevedere all'interno della Giunta una delega al commercio} in modo che l'Amministrazione lavori in modo stretto e sinergico con l'Associazione degli Operatori Economici di Dalmine (OPEC), per ravvivare il commercio di vicinato attraverso: eventi, formazione sull'innovazione, anagrafe dei fornitori e co-marketing;

\fakeparagraph{Tutela attiva e rilancio del commercio locale a livello urbanistico}, favorendo ad esempio l'imprenditorialità giovanile, anche grazie ai bandi di finanziamento dei Distretti del Commercio, mappando i locali sfitti e frenando la creazione di nuove aree dedicate a medie e grandi strutture di vendita;

\fakeparagraph{Riduzione della pressione fiscale sui commercianti locali} con sgravi e agevolazioni anche per nuove aperture e per specifiche categorie di esercizi (e.g. servizi per universitari, servizi essenziali per le frazioni, attività sostenibili e virtuose, imprenditoria giovanile);

\fakeparagraph{Politiche viabilistiche che favoriscano la fruibilità} ed aumentino l'affluenza verso gli esercizi del commercio di vicinato: recupero di piazze e centri storici, creazione di zone accessibili riservate ai pedoni (e.g. nei weekend) e reti ciclabili dedicate, insieme ad un maggiore controllo dei parcheggi a disco orario;

\fakeparagraph{Istituzione di un osservatorio del commercio}, ossia un tavolo comunale o sovracomunale che dovrà occuparsi di mappare le criticità, individuare soluzioni, creare un portale del commercio locale, ed agevolare i commercianti ed i futuri commercianti nella partecipazione a bandi e iniziative;

\fakeparagraph{Valutazione di spazi di coworking} per liberi professionisti, lavoratori da remoto e micro-imprese, preferendo la rigenerazione di un edificio già presente sul territorio;

\fakeparagraph{Promozione dei mercati di frazione temporanei in tutta la città}, in modo che i cittadini possano usufruirne almeno una volta a settimana.
