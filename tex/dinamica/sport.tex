% !TeX root = ../../main.tex
\section{Sport}
Lo sport e l'attività fisica rappresentano una formidabile occasione di crescita per i cittadini di tutte le età, nonché un veicolo di valori quali la tutela della salute, l'educazione e l'importanza della socialità. Attraverso le azioni previste e con una specifica delega all'interno del Consiglio Comunale, speriamo di poter incentivare i cittadini, a partire dai bambini fino agli adulti ed agli anziani, alla pratica abitudinaria dello sport. Vogliamo:

\fakeparagraph{Rafforzare il ruolo del Consorzio Polisportiva Dalmine} come collegamento tra l'Amministrazione e tutte le società sportive del territorio e di coordinamento per l'utilizzo degli impianti sportivi. Al contempo il Comune deve mantenere un canale di dialogo aperto con tutte le singole società sportive;

\fakeparagraph{Assicurare la manutenzione ordinaria e straordinaria di tutti gli impianti}, seguendo una pianificazione concordata con le società sportive che li utilizzano;

\fakeparagraph{Promuovere tutti gli sport} e soprattutto quelli considerati minori, anche attraverso una collaborazione tra pubblico e privato, così anche da portare nuovi sport ed eventi sportivi a Dalmine;

\fakeparagraph{Valorizzare le attività e le società sportive presenti nel Comune fornendo loro un supporto concreto}, dal punto di vista logistico e burocratico per l'accesso a bandi competitivi nazionali ed europei;

\begin{bluebox}
\fakeparagraph[]{Riconoscere lo sport come diritto di tutti i cittadini}, lavorando a diversi livelli:
\begin{itemize}
  \item con i servizi sociali per istituire borse sportive per permettere a nuclei familiari in difficoltà di accedere ad un fondo per le attività sportive;
  \item promuovendo lo sport inclusivo, con proposte rivolte agli atleti diversamente abili di tutte le età;
  \item incentivando la pratica sportiva tra i giovani e gli anziani, organizzando interventi educativi presso le scuole e sviluppando iniziative dedicate specificatamente alla terza età;
  \item facilitando i cittadini ad orientarsi nell'offerta attraverso la creazione di un portale di riferimento, fisico e/o digitale, in cui siano raccolte le varie attività sportive disponibili sul territorio;
\end{itemize}
\end{bluebox}

\fakeparagraph[]{Dotare il nostro comune di un centro sportivo} che diventi il fulcro delle manifestazioni e degli eventi sportivi, collegando meglio l'area che comprende Velodromo (che va reso più attrattivo e accessibile), piscine, campi da tennis ed il parco pubblico Pesenti di viale Locatelli, riqualificato in ottica sportiva. Intendiamo in sostanza portare avanti l'idea di una Cittadella dello Sport, valutando in futuro anche la realizzazione di un palazzetto;

\fakeparagraph{Ripensare parallelamente Dalmine come città dello sport diffuso}, installando nei parchi delle varie frazioni attrezzature sportive di libero utilizzo e spazi dedicati allo sport, come campi, attrezzature e percorsi;

\fakeparagraph{Dare vita ad una giornata comunale dello sport}, per valorizzare le società sportive e gli altri attori che promuovono lo sport sul territorio nonché gli atleti;

\fakeparagraph{Consentire la prenotazione degli impianti sportivi comunali a privati} semplificando la relativa burocrazia;

\fakeparagraph{Incrementare il numero dei defibrillatori automatici esterni} (DAE) presenti sul territorio, assicurando la loro manutenzione e istituendo delle giornate di formazione aperte al pubblico;

\fakeparagraph{Mantenere e rafforzare i rapporti con il Centro Universitario Sportivo} (CUS), per ampliare l'offerta sportiva a prezzi agevolati e attuare collaborazioni con le piscine;

\fakeparagraph{Garantire il rinnovamento delle piscine di Dalmine}, pianificando ulteriori opere di manutenzione e riqualificazione sia da un punto di vista architettonico che energetico al fine di massimizzare l'utilizzo e la fruibilità dell'impianto. Ciò comprende necessariamente un miglioramento del dialogo tra la società sportiva Onda Blu e l'Amministrazione al fine di garantire un lavoro in sinergia sulle varie tematiche.
