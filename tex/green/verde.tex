% !TeX root = ../../main.tex
\section{Verde Urbano}
\fakeparagraph[]{Realizzare un Piano comunale delle aree verdi}, mappando il patrimonio arboreo al fine di migliorarne la gestione e la manutenzione, in modo che sia preservato a livello numerico e che sia sempre in salute e non comprometta l'accessibilità della città;

\fakeparagraph{Creare nuove aree verdi e boschi urbani, e preservare quelle esistenti} (a livello di numero e di manutenzione) anche nell'ottica del contrasto all'emergenza caldo;  

\fakeparagraph{Incentivare la piantumazione su terreni privati o la riconversione di questi ultimi in aree verdi}, in particolare nelle zone vicine alle grandi arterie stradali, valutando la possibilità di fornire sgravi fiscali a chi decide di perseguire questa strada; 

\fakeparagraph{Riqualificare i parchi cittadini} equipaggiandoli di servizi igienici, fontanelle, panchine, cestini per la raccolta differenziata, giochi accessibili, attrezzatura sportiva e percorsi dedicati. Assicureremo una manutenzione costante dei parchi, rimuovendo eventuali barriere e coinvolgendo i cittadini nelle riqualificazioni e nelle iniziative svolte al loro interno; 

\fakeparagraph{Coinvolgere i Comuni facenti parte del PLIS e gli organi competenti per stabilire un divieto assoluto di attività venatoria nel parco}, essendo la caccia un'attività in contrasto con le finalità con cui il parco stesso è stato costituito e che compromette la frequentazione sicura da parte dei cittadini; 

\fakeparagraph{Vietare le attività legate al commercio ambulante nel PLIS}, sia permanenti che temporanee; 

\fakeparagraph{Riqualificare, tutelare e valorizzare il PLIS}, prolungando e riqualificando i sentieri ciclopedonali, installando fontanelle e panchine, riqualificando le infrastrutture presenti (e.g. impianti di depurazione) ed evitando interventi antropici eccessivi che limitino lo sviluppo e la preservazione della flora e della fauna. Trovare una soluzione sostenibile per la carenza di acqua presso l'oasi del Picchio Verde; 

\fakeparagraph{Creare una mappa di tutti i sentieri verdi e percorsi} del territorio, consultabile online e su tabelloni fisici. Aumentare la manutenzione dei sentieri e renderli più segnalati, riconoscibili e sicuri;

\fakeparagraph{Tutelare la biodiversità} sviluppando aree di natura selvaggia e incontaminata.
