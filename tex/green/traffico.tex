% !TeX root = ../../main.tex
\section{Traffico}
\fakeparagraph[]{Limitare il traffico veicolare in tutti i centri storici e le piazze}, che devono inoltre essere riqualificate e diventare spazi di aggregazioni vivibili, salubri e sicuri per la cittadinanza;

\fakeparagraph{Ridurre il traffico di attraversamento della città e le lunghe code interne}, favorendo il passaggio dei veicoli all'esterno con una regolamentazione efficace dell'incrocio tra via Provinciale e via Roma, e riqualificando i crocevia più importanti come ad esempio via Manzoni;

\fakeparagraph{Creare nuove zone a 30 km/h}, rendendo più sicuro il transito di auto, ciclisti e pedoni nelle zone critiche, e favorendo una riduzione dell'inquinamento atmosferico. Allo stesso tempo, rimarranno zone di scorrimento a 50 km/h nelle arterie principali di collegamento tra le frazioni e verso gli altri Comuni, ripensate perché sia conveniente e sicuro percorrerle in auto; 

\fakeparagraph{Istituire Zone a Traffico Limitato} nelle ore di punta così da limitare la circolazione dei veicoli nelle frazioni, soprattutto nelle zone nevralgiche percorse da pedoni e ciclisti, dirottando i non residenti preferibilmente sul Provinciale o su grandi viali esterni o di raccordo;

\fakeparagraph{Contrastare l'autostrada Bergamo - Treviglio}, un'opera dannosa per il territorio che sottrarrà altre aree verdi e spianerà la strada ai grandi poli logistici. Si tratta di un progetto anacronistico, finanziato da ingenti fondi pubblici e dimostratosi incapace di ridurre il traffico tra le città che dovrebbe collegare. In maniera sinergica con gli altri Comuni contrari assicureremo pieno sostegno ai comitati e alle associazioni che si oppongono all'opera oltre che la promozione di soluzioni alternative, quali il potenziamento del trasporto pubblico locale e lo sblocco delle piccole opere che permetterebbero la fluidificazione del traffico provinciale;

\fakeparagraph{Mettere in sicurezza strade ed infrastrutture} in tutte le frazioni attraverso la loro manutenzione costante, la regolamentazione della viabilità nei tratti e negli incroci più critici, l'incremento e l'automatizzazione dei controlli riguardanti il rispetto dei limiti di velocità e l'installazione di dissuasori non pericolosi;

\fakeparagraph{Disincentivare la sosta presso le scuole} tramite chiusura temporanea delle strade e istituire dei divieti di sosta e fermata in alcune vie prive di marciapiedi, in modo da non ridurre lo spazio disponibile per pedoni e ciclisti;

\fakeparagraph{Portare all'esterno della città i grandi parcheggi} sfruttando le aree edificate già presenti e lavorando in profondità, così da ridurre il passaggio nelle frazioni e nel centro di chi si reca in automobile a Dalmine. Allo stesso tempo, è necessario offrire agli utenti servizi alternativi all'automobile per raggiungere la relativa destinazione finale (e.g. bike sharing, navetta verso i luoghi d'interesse, trasporto pubblico efficiente);

\fakeparagraph{Regolamentare la sosta selvaggia dei mezzi pesanti}, ampliando e riqualificando il parcheggio di via dell'Artigianato, dotandolo ad esempio di bagni pubblici. Ipotizzare la creazione di un parcheggio dedicato fuori dalla città insieme ai comuni limitrofi, preferibilmente in una zona già urbanizzata; 

\fakeparagraph{Vietare la circolazione ai mezzi motorizzati nelle zone del Belvedere e dell'Oasi naturalistica}, consentendo l'accesso e la sosta solo ai residenti, ai clienti ed agli operatori delle attività commerciali e alle persone con disabilità;

\fakeparagraph{Ridurre l'impatto dei lavori pubblici sul traffico e sulla sicurezza delle strade}, coordinando e pianificando meglio i vari interventi sul territorio ed interfacciandosi più efficacemente con le ditte appaltatrici ed i progettisti in merito a risultati e tempistiche.
