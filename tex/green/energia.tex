% !TeX root = ../../main.tex
\section[Energia, Inquinamento, Rifiuti]{Energia, Inquinamento,\\Rifiuti}
\fakeparagraph[]{Contrastare l'inquinamento atmosferico} raggiungendo gli obiettivi regionali, incrementando il controllo ed il monitoraggio dei soggetti inquinanti e redigendo un Piano operativo per la qualità dell'aria; 

\fakeparagraph{Contrastare la siccità e il dissesto idrogeologico} attraverso un piano di comunicazione contro gli sprechi idrici ed ordinanze mirate. Sollecitare il completamento degli interventi di manutenzione dei bacini e di tutti i corsi d'acqua previsti dal PNRR e attuare misure a contrasto della dispersione idrica;

\begin{bluebox}
\fakeparagraph[]{Sviluppare politiche volte al risparmio energetico ed alla sostenibilità} sia in ambito pubblico sia privato. Ciò potrebbe ricomprendere:
\begin{itemize}
  \item la revisione dell'illuminazione di strade ed edifici comunali e del relativo riscaldamento;
  \item la creazione di uno Sportello Energia che si occupi tra le altre cose di un'opportuna comunicazione ai privati sul tema; 
  \item la promozione delle Comunità Energetiche Rinnovabili e il consolidamento di una Comunità Energetica Comunale, che possa favorire le famiglie in condizione di povertà energetica;  
  \item rilanciare il teleriscaldamento, tutelando gli attuali fruitori del servizio, rivedendo la convenzione con Tenaris per fissare limiti e tetti, prevedendo soglie non superabili in condizioni emergenziali ed evitando che il Comune si arricchisca con le royalties derivanti, ridistribuendole invece agli utenti del teleriscaldamento; 
\end{itemize} 
\end{bluebox}

\fakeparagraph[]{Proseguire verso l'obiettivo ``rifiuti zero''}, con una strategia che minimizzi i rifiuti e massimizzi il riciclo, con il miglioramento della percentuale e della qualità della raccolta differenziata fino al completamento della tariffa puntuale e sensibilizzando i cittadini sul consumo sostenibile (e.g. riuso, seconda mano, scambio);

\fakeparagraph{Realizzare la Cittadella del Riuso}, uno spazio sicuro presso la piattaforma ecologica dove trovare e scambiare gratuitamente oggetti in buono stato, con la presenza di personale dedicato. Promuovere e patrocinare realtà ed iniziative volte al ricircolo degli oggetti;

\fakeparagraph{Porre attenzione verso il decoro urbano}, assicurando una presenza più capillare di cestini, posacenere e distributori di sacchetti per deiezioni canine, unita ad una forte comunicazione che disincentivi l'abbandono di rifiuti, anche attraverso l'installazione di fototrappole in punti strategici. Incentivare le segnalazioni tramite Municipium;

\fakeparagraph{Monitorare il rispetto degli standard di qualità da parte del servizio di raccolta rifiuti e pulizia urbana}, anche rivedendo il regolamento della Piattaforma Ecologica; 

\fakeparagraph{Contrastare l'inquinamento acustico}, cercando di ridurre il rumore delle attività produttive nelle zone residenziali, mitigando il rumore derivante dal traffico stradale grazie alla  piantumazione di alberature apposite, l'utilizzo di materiali fonoassorbenti e assicurando il rispetto di criteri di distanza minimi;

\fakeparagraph{Inserire il Comune di Dalmine nel tavolo di lavoro relativo all'Aeroporto di Orio al Serio} al fine di implementare rotte che provochino meno disagi acustici, e promuovendo l'installazione di soluzioni di isolamento acustico;

\fakeparagraph{Aumentare la presenza e la capillarità dei distributori pubblici d'acqua}, incentivando privati, aziende ed esercizi del territorio affinché si dotino di distributori riducendo la plastica monouso; 

\fakeparagraph{Contrastare la presenza di amianto} attraverso una mappatura completa della sua presenza sul territorio, intensificando il lavoro di informazione e comunicazione in merito e lavorando affinché si possano proporre tariffe agevolate per la rimozione; 

\fakeparagraph{Potenziare l'organico dell'Ufficio Ecologia e le competenze del personale} qui impiegato in modo tale che abbia una visione più ampia di transizione ecologica, ponendo attenzione ad iniziative e bandi nazionali ed internazionali sul tema; 

\fakeparagraph{Realizzare progetti di educazione ambientale} così da promuovere il rispetto degli ecosistemi, la conoscenza della natura e l'importanza dei comportamenti virtuosi nel contrasto al cambiamento climatico; 

\fakeparagraph{Prevedere progetti contro lo spreco alimentare e di sostegno al consumo consapevole}, sostenibile e solidale, coinvolgendo anche le attività commerciali e le mense scolastiche;

\fakeparagraph{Istituire convenzioni per la realizzazione di opere di compensazione ambientali} serie, tangibili e vantaggiose per la cittadinanza in ogni rapporto con grandi aziende e soggetti potenzialmente inquinanti;

\fakeparagraph{Assicurare il monitoraggio e la trasparenza} attraverso la creazione di una pagina sul sito del Comune che faccia da aggregatore rispetto a tutti i database sulle emissioni e su ogni forma d'inquinamento;

\fakeparagraph{Promuovere l'interramento delle linee di trasmissione} elettrica aeree ad alta e altissima tensione tramite l'interlocuzione con le società coinvolte.
