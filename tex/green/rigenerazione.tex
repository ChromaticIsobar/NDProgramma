% !TeX root = ../../main.tex
\section{Rigenerazione Urbana}
\fakeparagraph[]{Cambiare la destinazione d'uso dei terreni} comunali inseriti nel Piano delle alienazioni destinandoli a parchi pubblici, frutteti e orti sociali curati dai cittadini, contro l'urbanizzazione selvaggia del territorio e il consumo di suolo ma sempre nel rispetto degli equilibri finanziari;

\fakeparagraph{Rivedere il Piano di Governo del Territorio} in modo che gli attuali piani attuativi non prevedano un'eccessiva cementificazione a scapito delle aree verdi;

\fakeparagraph{Riguadagnare il territorio}, attuando tutte le disposizioni che consentano ai cittadini che possiedono un terreno edificabile di poter retrocedere a terreno agricolo, in ottemperanza alla Legge Regionale sul consumo di suolo; 

\fakeparagraph{Intervenire sulla tassazione locale per ridurre ulteriormente il carico fiscale sulle aree agricole}, che vanno mantenute e salvaguardate;

\fakeparagraph{Mappare e riconvertire i luoghi cementificati superflui} e trasformarli in spazi verdi o di pubblica utilità, con modalità dal basso impatto ecologico; 

\fakeparagraph{Mappare e riqualificare gli edifici e le aree degradati e abbandonati} sia pubblici sia privati, intervenendo sulla tassazione locale nei confronti di chi ha immobili non utilizzati e sfruttando maggiormente gli incentivi esistenti per favorire la rigenerazione urbana prima di procedere con nuove edificazioni, che in ogni caso dovrebbero essere preferibilmente sviluppate in verticale e non in orizzontale;

\fakeparagraph{Utilizzare in modo mirato gli oneri di urbanizzazion}e incassati dai piani attuativi, impiegandoli per mitigare l'impatto delle costruzioni da cui derivano e dialogare con gli attuatori affinché realizzino contestualmente alle costruzioni opere di pubblica utilità (e.g. parcheggi residenziali);

\fakeparagraph{Creare un tavolo con i costruttori ed altri attori importanti del territorio} per identificare le  migliori soluzioni urbanistiche ed ambientali.
