% !TeX root = ../../main.tex
\section{Trasporto Pubblico}
\fakeparagraph[]{Stimolare la fruizione del trasporto pubblico} attraverso un'interlocuzione costante e costruttiva con tutte le aziende del Trasporto Pubblico Locale (TPL), in modo che diventi un'alternativa efficiente all'utilizzo di un mezzo proprio. A tale scopo si dovrebbe prevedere l'installazione di pensiline e panchine ad ogni fermata, di totem digitali informativi con orari e tratte in tempo reale oltre alla disponibilità di opzioni fisiche e digitali più diffuse per l'acquisto dei biglietti;

\fakeparagraph{Migliorare il collegamento con Bergamo e tra le frazioni di Dalmine}, anche in ottica universitaria. Bisogna quindi incrementare il numero delle fermate e rendere più frequenti, in modo particolare nelle frazioni esterne, le corse in tarda serata, di prima mattina e nel fine settimana; 

\fakeparagraph{Migliorare il collegamento con la stazione ferroviaria di Verdello} assicurando la coincidenza con treni in partenza/arrivo ma anche con quelle di Levate e Stezzano;

\fakeparagraph{Valorizzare il progetto di linea elettrica veloce E-BRT}, che collegherà Bergamo, Dalmine e la stazione ferroviaria di Verdello creando una nuova soluzione per i pendolari. Allo stesso tempo sarà necessario vigilare l'impatto dell'opera sulla viabilità locale;

\fakeparagraph{Riqualificare le aree delle fermate dell'autobus autostradale} (Z301) in collaborazione con gli organi competenti. 
