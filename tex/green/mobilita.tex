% !TeX root = ../../main.tex
\section{Mobilità Sostenibile}
\fakeparagraph[]{Migliorare tutti i percorsi e le aree pedonali} realizzando attraversamenti pedonali sicuri, illuminati, ben segnalati e rialzati, lontani da parcheggi e passi carrabili, implementando più semafori a chiamata per i pedoni, e mappando lo stato di marciapiedi e delle rampe di accesso per poi riqualificare quelli maggiormente degradati;

\fakeparagraph[]{Collegare in modo veloce e sicuro Sabbio e Guzzanica} con il resto della città alla luce della necessità di attraversare l'ex SP 525;

\fakeparagraph{Realizzare piste ciclopedonali sicure e funzionali}, assicurandone la continuità, l'uniformità e la riconoscibilità, oltre che un'adeguata separazione dalla sede stradale. Garantire uno stanziamento certo ogni anno per realizzare la ciclopolitana, ovvero una vera e propria rete di piste ciclabili che colleghino punti strategici della città ed in particolare le scuole;

\fakeparagraph{Creare servizi accessori alla ciclabilità}, come parcheggi e depositi (e.g. in Piazza Risorgimento), rastrelliere, totem di riparazione, punti di ricarica per e-bike e monopattini elettrici;

\fakeparagraph{Creare collegamenti rapidi tra piste ciclabili}, fermate del trasporto pubblico e grandi parcheggi;

\fakeparagraph{Aumentare e curare i percorsi ciclopedonali verso tutti i Comuni limitrofi}, di concerto con questi ultimi ed alla luce del protocollo d'intesa esistente;

\fakeparagraph{Implementare servizi di sharing sostenibili} per biciclette, e-bike e monopattini elettrici;

\fakeparagraph{Realizzare un progetto comunale di car pooling}, incentivando lo spostamento condiviso verso aziende, enti pubblici e scuole;

\fakeparagraph{Incrementare il numero di colonnine per la ricarica} di veicoli elettrici/ibridi e di parcheggi riservati, anche come opere di compensazione, e proporre convenzioni che possano rendere più economico il loro utilizzo; 

\fakeparagraph{Realizzare progetti di sensibilizzazione} e campagne informative riguardo la mobilità sostenibile;

\fakeparagraph{Promuovere l'acquisto di altre auto elettriche dedicate ai dipendenti comunali ed alle Forze dell'Ordine} durante gli orari di lavoro e a tutti i cittadini nel resto della giornata, sul modello di un progetto di car sharing.
