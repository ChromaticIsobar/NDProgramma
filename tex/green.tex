% !TeX root = ../main.tex
\chapter{Dalmine +Green}
La nostra visione di Dalmine è quella di una città che sposa gli ideali dello sviluppo sostenibile e circolare e della mobilità dolce, preservando così l'ambiente in cui si inserisce e migliorando la salute e la qualità della vita dei suoi cittadini. L'ambiente sarà al centro della nostra azione amministrativa, per trasformare la Dalmine di oggi: una città troppo calda, in cui i limiti sul particolato atmosferico non sono rispettati, in cui il 55\% del territorio è cementificato e la dispersione idrica sfiora il 50\%. A tal fine abbiamo individuato alcuni punti cardine su cui lavorare che sono qui di seguito riassunti ma verranno dettagliati nelle sezioni successive:

\fakeparagraph{Ridurre il traffico, rendendo la città più sicura e a misura di pedoni e ciclisti};

\fakeparagraph{Rivisitare i piani urbanistici e attuativi per renderli sostenibili e armonizzati} dando priorità ad interventi di pubblica utilità;

\fakeparagraph{Dare priorità alla rigenerazione dell'esistente sulla costruzione del nuovo}, contro il consumo di suolo selvaggio e lo spreco di terreno, censendo ad esempio gli stabili vuoti, in disuso e con potenzialità inespresse;

\fakeparagraph{Stilare un Patto per l'Ambiente} di ampio respiro che coinvolga istituzioni, enti, scuole, aziende, cittadini, associazioni e volontari e che definisca obiettivi ed azioni concrete a sua tutela.

% !TeX root = ../../main.tex
\section{Traffico}

% !TeX root = ../../main.tex
\section{Mobilità Sostenibile}

% !TeX root = ../../main.tex
\section{Trasporto Pubblico}
\fakeparagraph[]{Stimolare la fruizione del trasporto pubblico} attraverso un'interlocuzione costante e costruttiva con tutte le aziende del Trasporto Pubblico Locale (TPL), in modo che diventi un'alternativa efficiente all'utilizzo di un mezzo proprio. A tale scopo si dovrebbe prevedere l'installazione di pensiline e panchine ad ogni fermata, di totem digitali informativi con orari e tratte in tempo reale oltre alla disponibilità di opzioni fisiche e digitali più diffuse per l'acquisto dei biglietti;

\fakeparagraph{Migliorare il collegamento con Bergamo e tra le frazioni di Dalmine}, anche in ottica universitaria. Bisogna quindi incrementare il numero delle fermate e rendere più frequenti, in modo particolare nelle frazioni esterne, le corse in tarda serata, di prima mattina e nel fine settimana; 

\fakeparagraph{Migliorare il collegamento con la stazione ferroviaria di Verdello} assicurando la coincidenza con treni in partenza/arrivo ma anche con quelle di Levate e Stezzano;

\fakeparagraph{Valorizzare il progetto di linea elettrica veloce E-BRT}, che collegherà Bergamo, Dalmine e la stazione ferroviaria di Verdello creando una nuova soluzione per i pendolari. Allo stesso tempo sarà necessario vigilare l'impatto dell'opera sulla viabilità locale;

\fakeparagraph{Riqualificare le aree delle fermate dell'autobus autostradale} (Z301) in collaborazione con gli organi competenti. 

% !TeX root = ../../main.tex
\section{Rigenerazione Urbana}
\fakeparagraph[]{Cambiare la destinazione d'uso dei terreni} comunali inseriti nel Piano delle alienazioni destinandoli a parchi pubblici, frutteti e orti sociali curati dai cittadini, contro l'urbanizzazione selvaggia del territorio e il consumo di suolo ma sempre nel rispetto degli equilibri finanziari;

\fakeparagraph{Rivedere il Piano di Governo del Territorio} in modo che gli attuali piani attuativi non prevedano un'eccessiva cementificazione a scapito delle aree verdi;

\fakeparagraph{Riguadagnare il territorio}, attuando tutte le disposizioni che consentano ai cittadini che possiedono un terreno edificabile di poter retrocedere a terreno agricolo, in ottemperanza alla Legge Regionale sul consumo di suolo; 

\fakeparagraph{Intervenire sulla tassazione locale per ridurre ulteriormente il carico fiscale sulle aree agricole}, che vanno mantenute e salvaguardate;

\fakeparagraph{Mappare e riconvertire i luoghi cementificati superflui} e trasformarli in spazi verdi o di pubblica utilità, con modalità dal basso impatto ecologico; 

\fakeparagraph{Mappare e riqualificare gli edifici e le aree degradati e abbandonati} sia pubblici sia privati, intervenendo sulla tassazione locale nei confronti di chi ha immobili non utilizzati e sfruttando maggiormente gli incentivi esistenti per favorire la rigenerazione urbana prima di procedere con nuove edificazioni, che in ogni caso dovrebbero essere preferibilmente sviluppate in verticale e non in orizzontale;

\fakeparagraph{Utilizzare in modo mirato gli oneri di urbanizzazione} incassati dai piani attuativi, impiegandoli per mitigare l'impatto delle costruzioni da cui derivano e dialogare con gli attuatori affinché realizzino contestualmente alle costruzioni opere di pubblica utilità (e.g. parcheggi residenziali);

\fakeparagraph{Creare un tavolo con i costruttori ed altri attori importanti del territorio} per identificare le  migliori soluzioni urbanistiche ed ambientali.

% !TeX root = ../../main.tex
\section{Verde Urbano}
\fakeparagraph[]{Realizzare un Piano comunale delle aree verdi}, mappando il patrimonio arboreo al fine di migliorarne la gestione e la manutenzione, in modo che sia preservato a livello numerico e che sia sempre in salute e non comprometta l'accessibilità della città;

\fakeparagraph{Creare nuove aree verdi e boschi urbani, e preservare quelle esistenti} (a livello di numero e di manutenzione) anche nell'ottica del contrasto all'emergenza caldo;  

\fakeparagraph{Incentivare la piantumazione su terreni privati o la riconversione di questi ultimi in aree verdi}, in particolare nelle zone vicine alle grandi arterie stradali, valutando la possibilità di fornire sgravi fiscali a chi decide di perseguire questa strada; 

\fakeparagraph{Riqualificare i parchi cittadini} equipaggiandoli di servizi igienici, fontanelle, panchine, cestini per la raccolta differenziata, giochi accessibili, attrezzatura sportiva e percorsi dedicati. Assicureremo una manutenzione costante dei parchi, rimuovendo eventuali barriere e coinvolgendo i cittadini nelle riqualificazioni e nelle iniziative svolte al loro interno; 

\fakeparagraph{Coinvolgere i Comuni facenti parte del PLIS e gli organi competenti per stabilire un divieto assoluto di attività venatoria nel parco}, essendo la caccia un'attività in contrasto con le finalità con cui il parco stesso è stato costituito e che compromette la frequentazione sicura da parte dei cittadini; 

\fakeparagraph{Vietare le attività legate al commercio ambulante nel PLIS}, sia permanenti che temporanee; 

\fakeparagraph{Riqualificare, tutelare e valorizzare il PLIS}, prolungando e riqualificando i sentieri ciclopedonali, installando fontanelle e panchine, riqualificando le infrastrutture presenti (e.g. impianti di depurazione) ed evitando interventi antropici eccessivi che limitino lo sviluppo e la preservazione della flora e della fauna. Trovare una soluzione sostenibile per la carenza di acqua presso l'oasi del Picchio Verde; 

\fakeparagraph{Creare una mappa di tutti i sentieri verdi e percorsi} del territorio, consultabile online e su tabelloni fisici. Aumentare la manutenzione dei sentieri e renderli più segnalati, riconoscibili e sicuri;

\fakeparagraph{Tutelare la biodiversità} sviluppando aree di natura selvaggia e incontaminata.

% !TeX root = ../../main.tex
\section[Energia, Inquinamento, Rifiuti]{Energia, Inquinamento,\\Rifiuti}

