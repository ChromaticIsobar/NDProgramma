% !TeX root = ../main.tex
\chapter{Dalmine +Green}
La nostra visione di Dalmine è quella di una città che sposa gli ideali dello sviluppo sostenibile e circolare e della mobilità dolce, preservando così l'ambiente in cui si inserisce e migliorando la salute e la qualità della vita dei suoi cittadini. L'ambiente sarà al centro della nostra azione amministrativa, per trasformare la Dalmine di oggi: una città troppo calda, in cui i limiti sul particolato atmosferico non sono rispettati, in cui il 55\% del territorio è cementificato e la dispersione idrica sfiora il 50\%. A tal fine abbiamo individuato alcuni punti cardine su cui lavorare, qui di seguito riassunti e dettagliati nelle sezioni successive:

\fakeparagraph{Ridurre il traffico, rendendo la città più sicura e a misura di pedoni e ciclisti};

\fakeparagraph{Rivisitare i piani urbanistici e attuativi per renderli sostenibili e armonizzati} dando priorità ad interventi di pubblica utilità;

\fakeparagraph{Dare priorità alla rigenerazione dell'esistente sulla costruzione del nuovo}, contro il consumo di suolo selvaggio e lo spreco di terreno, censendo ad esempio gli stabili vuoti, in disuso e con potenzialità inespresse;

\fakeparagraph{Stilare un Patto per l'Ambiente} di ampio respiro che coinvolga istituzioni, enti, scuole, aziende, cittadini e associazioni e che definisca obiettivi ed azioni concrete a sua tutela.

% !TeX root = ../../main.tex
\section{Traffico}
\fakeparagraph[]{Limitare il traffico veicolare in tutti i centri storici e le piazze}, che devono inoltre essere riqualificate e diventare spazi di aggregazioni vivibili, salubri e sicuri per la cittadinanza;

\fakeparagraph{Ridurre il traffico di attraversamento della città e le lunghe code interne}, favorendo il passaggio dei veicoli all'esterno con una regolamentazione efficace dell'incrocio tra via Provinciale e via Roma, e riqualificando i crocevia più importanti come ad esempio via Manzoni;

\fakeparagraph{Creare nuove zone a 30 km/h}, rendendo più sicuro il transito di auto, ciclisti e pedoni nelle zone critiche, e favorendo una riduzione dell'inquinamento atmosferico. Allo stesso tempo, rimarranno zone di scorrimento a 50 km/h nelle arterie principali di collegamento tra le frazioni e verso gli altri Comuni, ripensate perché siano convenienti e sicure; 

\begin{bluebox}
\fakeparagraph[]{Contrastare l'autostrada Bergamo - Treviglio}, un'opera dannosa per il territorio che sottrarrà altre aree verdi e spianerà la strada ai grandi poli logistici. Si tratta di un progetto anacronistico, finanziato da ingenti fondi pubblici e dimostratosi incapace di ridurre il traffico tra le città che dovrebbe collegare. In maniera sinergica con gli altri Comuni contrari assicureremo pieno sostegno ai comitati e alle associazioni che si oppongono all'opera oltre che la promozione di soluzioni alternative, quali il potenziamento del trasporto pubblico locale e lo sblocco delle piccole opere che permetterebbero la fluidificazione del traffico provinciale;
\end{bluebox}

\fakeparagraph[]{Disincentivare la sosta presso le scuole} tramite chiusura temporanea delle strade e istituire dei divieti di sosta e fermata in alcune vie prive di marciapiedi, in modo da non ridurre lo spazio disponibile per pedoni e ciclisti;

\fakeparagraph[]{Mettere in sicurezza strade ed infrastrutture} in tutte le frazioni attraverso la loro manutenzione costante, la regolamentazione della viabilità nei tratti e negli incroci più critici, l'incremento e l'automatizzazione dei controlli riguardanti il rispetto dei limiti di velocità e l'installazione di dissuasori non pericolosi;

\fakeparagraph{Portare all'esterno della città i grandi parcheggi} sfruttando le aree edificate già presenti e lavorando in profondità, così da ridurre il passaggio nelle frazioni e nel centro di chi si reca in automobile a Dalmine. Allo stesso tempo, è necessario offrire agli utenti servizi alternativi all'automobile per raggiungere la relativa destinazione finale (e.g. bike sharing, navetta verso i luoghi d'interesse, trasporto pubblico efficiente);

\fakeparagraph{Regolamentare la sosta selvaggia dei mezzi pesanti}, ampliando e riqualificando il parcheggio di via dell'Artigianato, dotandolo ad esempio di bagni pubblici. Ipotizzare la creazione di un parcheggio dedicato fuori dalla città insieme ai comuni limitrofi, preferibilmente in una zona già urbanizzata; 

\fakeparagraph{Vietare la circolazione ai mezzi motorizzati nelle zone del Belvedere e dell'Oasi naturalistica}, consentendo l'accesso e la sosta solo ai residenti, ai clienti ed agli operatori delle attività commerciali e alle persone con disabilità;

\fakeparagraph{Ridurre l'impatto dei lavori pubblici sul traffico e sulla sicurezza delle strade}, coordinando e pianificando meglio i vari interventi sul territorio ed interfacciandosi più efficacemente con le ditte appaltatrici ed i progettisti in merito a risultati e tempistiche;

\begin{bluebox}
\fakeparagraph[]{Istituire Zone a Traffico Limitato} nelle ore di punta così da limitare la circolazione dei veicoli nelle frazioni, soprattutto nelle zone nevralgiche percorse da pedoni e ciclisti, dirottando i non residenti preferibilmente sul Provinciale o su grandi viali esterni o di raccordo.
\end{bluebox}

% !TeX root = ../../main.tex
\section{Mobilità Sostenibile}
\fakeparagraph[]{Migliorare tutti i percorsi e le aree pedonali} realizzando attraversamenti pedonali sicuri, illuminati, ben segnalati e rialzati, lontani da parcheggi e passi carrabili, implementando più semafori a chiamata per i pedoni, e mappando lo stato di marciapiedi e delle rampe di accesso per poi riqualificare quelli maggiormente degradati;

\fakeparagraph[]{Collegare in modo veloce e sicuro Sabbio e Guzzanica} con il resto della città alla luce della necessità di attraversare l'ex SP 525;

\fakeparagraph{Realizzare piste ciclopedonali sicure e funzionali}, assicurandone la continuità, l'uniformità e la riconoscibilità, oltre che un'adeguata separazione dalla sede stradale. Garantire uno stanziamento certo ogni anno per realizzare la ciclopolitana, ovvero una vera e propria rete di piste ciclabili che colleghino punti strategici della città ed in particolare le scuole;

\fakeparagraph{Creare servizi accessori alla ciclabilità}, come parcheggi e depositi (e.g. in Piazza Risorgimento), rastrelliere, totem di riparazione, punti di ricarica per e-bike e monopattini elettrici;

\fakeparagraph{Creare collegamenti rapidi tra piste ciclabili}, fermate del trasporto pubblico e grandi parcheggi;

\fakeparagraph{Aumentare e curare i percorsi ciclopedonali verso tutti i Comuni limitrofi}, di concerto con questi ultimi ed alla luce del protocollo d'intesa esistente;

\fakeparagraph{Implementare servizi di sharing sostenibili} per biciclette, e-bike e monopattini elettrici;

\fakeparagraph{Realizzare un progetto comunale di car pooling}, incentivando lo spostamento condiviso verso aziende, enti pubblici e scuole;

\fakeparagraph{Incrementare il numero di colonnine per la ricarica} di veicoli elettrici/ibridi e di parcheggi riservati, anche come opere di compensazione, e proporre convenzioni che possano rendere più economico il loro utilizzo; 

\fakeparagraph{Realizzare progetti di sensibilizzazione} e campagne informative riguardo la mobilità sostenibile;

\fakeparagraph{Promuovere l'acquisto di altre auto elettriche dedicate ai dipendenti comunali ed alle Forze dell'Ordine} durante gli orari di lavoro e a tutti i cittadini nel resto della giornata, sul modello di un progetto di car sharing.

% !TeX root = ../../main.tex
\section{Trasporto Pubblico}

% !TeX root = ../../main.tex
\section{Rigenerazione Urbana}
\fakeparagraph[]{Cambiare la destinazione d'uso dei terreni} comunali inseriti nel Piano delle alienazioni destinandoli a parchi pubblici, frutteti e orti sociali curati dai cittadini, contro l'urbanizzazione selvaggia del territorio e il consumo di suolo ma sempre nel rispetto degli equilibri finanziari;

\fakeparagraph{Rivedere il Piano di Governo del Territorio} in modo che gli attuali piani attuativi non prevedano un'eccessiva cementificazione a scapito delle aree verdi;

\fakeparagraph{Riguadagnare il territorio}, attuando tutte le disposizioni che consentano ai cittadini che possiedono un terreno edificabile di poter retrocedere a terreno agricolo, in ottemperanza alla Legge Regionale sul consumo di suolo; 

\fakeparagraph{Intervenire sulla tassazione locale per ridurre ulteriormente il carico fiscale sulle aree agricole}, che vanno mantenute e salvaguardate;

\fakeparagraph{Mappare e riconvertire i luoghi cementificati superflui} e trasformarli in spazi verdi o di pubblica utilità, con modalità dal basso impatto ecologico; 

\fakeparagraph{Mappare e riqualificare gli edifici e le aree degradati e abbandonati} sia pubblici sia privati, intervenendo sulla tassazione locale nei confronti di chi ha immobili non utilizzati e sfruttando maggiormente gli incentivi esistenti per favorire la rigenerazione urbana prima di procedere con nuove edificazioni, che in ogni caso dovrebbero essere preferibilmente sviluppate in verticale e non in orizzontale;

\fakeparagraph{Utilizzare in modo mirato gli oneri di urbanizzazion}e incassati dai piani attuativi, impiegandoli per mitigare l'impatto delle costruzioni da cui derivano e dialogare con gli attuatori affinché realizzino contestualmente alle costruzioni opere di pubblica utilità (e.g. parcheggi residenziali);

\fakeparagraph{Creare un tavolo con i costruttori ed altri attori importanti del territorio} per identificare le  migliori soluzioni urbanistiche ed ambientali.

% !TeX root = ../../main.tex
\section{Verde Urbano}

% !TeX root = ../../main.tex
\section{Energia, Inquinamento, Rifiuti}

