% !TeX root = ../../main.tex
\section{Stile di Amministrazione}
Lo stile di Amministrazione è per noi importante quanto le linee programmatiche. Saremo un'Amministrazione trasparente, capace di interfacciarsi in modo costruttivo con gli altri gruppi politici, con i maggiori attori del territorio e con i Comuni vicini e di essere un punto di riferimento per l'intera macchina comunale. Intendiamo riavvicinare la cittadinanza alle istituzioni, facendola partecipare sia nei processi decisionali sia nelle fasi progettuali. Vogliamo:

\fakeparagraph{Istituire strumenti di consultazione e partecipazione}, come momenti formali ed informali di incontro con i cittadini, assemblee pubbliche in tutte le frazioni, prove di “bilancio partecipativo” e sistemi di condivisione digitali;

\fakeparagraph{Garantire la consultazione di tutti i gruppi consiliari nella fase progettuale} di opere pubbliche di grande rilievo, investimenti importanti, redazione di regolamenti e piani attuativi;

\fakeparagraph{Approvare il calendario delle sedute del consiglio comunale ad inizio anno}, durante la conferenza dei capigruppo, così da massimizzare la partecipazione dei consiglieri e della cittadinanza;

\fakeparagraph{Promuovere incontri cadenzati con  il personale comunale} così da rilevare le opinioni, le critiche e i suggerimenti di chi vive l'ente quotidianamente;

\fakeparagraph{Potenziare gli uffici comunali}, ad esempio ipotizzando un ufficio sovracomunale dedicato alla stesura/gestione dei bandi europei;

\fakeparagraph{Incrementare il dialogo con i Comuni vicini} per progetti che possono influenzare i diversi enti positivamente;

\fakeparagraph{Migliorare la comunicazione} per renderla più diretta ed accessibile ai cittadini. Allo stesso tempo, ci assicureremo di essere più ricettivi verso le segnalazioni, garantendo un adeguato tasso di risposta.
