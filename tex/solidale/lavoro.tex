% !TeX root = ../../main.tex
\section{Lavoro e Contrasto alla Povertà}
\fakeparagraph{Facilitare l'orientamento al lavoro} potenziando i servizi già attivi sul territorio e stimolando il lavoro in sinergia dei Servizi Sociali e delle associazioni/cooperative attive in quest'ambito;

\fakeparagraph{Stabilire percorsi preferenziali per l'ingresso o il ritorno nel mondo del lavoro dei NEET}, con supporto pratico e costante e con la garanzia di impieghi dignitosi e regolari; 

\fakeparagraph{Redigere una guida comunale del lavoro}, un opuscolo informativo multilingua fisico e digitale che evidenzi le realtà attive sul territorio (e.g. sindacati, agenzie del lavoro, associazioni, corsi di formazione);

\fakeparagraph{Contrastare l'emarginazione} delle persone che non riescono a trovare un impiego e/o soffrono di un disagio psico-sociale, offrendo loro esperienze temporanee di volontariato con un equo rimborso spese;

\fakeparagraph{Costituire l'Alleanza locale contro la povertà} coinvolgendo i sindacati e le associazioni del territorio, per reperire finanziamenti per l'inserimento lavorativo delle persone fragili, incentivare tirocini e interventi riabilitativi;

\fakeparagraph{Favorire la formazione in ambito tecnologico} attraverso programmi e corsi gestiti da volontari, magari coinvolgendo gli studenti delle scuole superiori per favorire lo scambio intergenerazionale. Prevedere in generale investimenti a favore della formazione lavorativa dei cittadini;

\fakeparagraph{Fornire assistenza a piccole imprese, artigiani e lavoratori dipendenti} in merito alla  prevenzione dei rischi lavorativi coinvolgendo  Bergamo Sviluppo e Azienda Speciale della Camera di Commercio di Bergamo;

\fakeparagraph{Attivare il servizio ``Chiedilo al Notaio''}, un'attività di consulenza gratuita per i cittadini promossa dal Consiglio Notarile di Bergamo volta a dare risposte rapide a quesiti in materia civile, fiscale e successoria;

\fakeparagraph{Ridistribuire i surplus derivanti dall'istituzione di un sistema di tassazione equo e progressivo} (che ad esempio sostituisca l'addizionale IRPEF unica con un sistema a scaglioni e che non gravi sui più deboli) per progetti di carattere sociale;

\fakeparagraph{Definire misure strutturali a supporto delle famiglie} che sono state più toccate  dall'aumento dei costi delle utenze energetiche, dei beni di prima necessità e delle prestazioni mediche per esempio attraverso un tariffario progressivo dei servizi comunali in base all'ISEE e fondi dedicati;

\fakeparagraph{Istituire il Baratto Amministrativo}, un sistema che permette a chi fatica a pagare tasse ed imposte comunali di sdebitarsi con piccoli lavori per l'ente in cambio di sconti sul debito accumulato; 

\fakeparagraph{Promuovere l'educazione finanziaria} dei cittadini riguardo strumenti basilari di cura del proprio patrimonio. 
