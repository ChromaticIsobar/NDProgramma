% !TeX root = ../../main.tex
\section{Politiche Giovanili}
\fakeparagraph{Istituire una delega specifica alle politiche giovanili}, che possa essere conferita ad un assessore o ad un consigliere comunale;

\fakeparagraph{Riprogettare la Commissione Giovani} perché sia un tavolo di rappresentanza politica e del territorio e costituisca il luogo primario per la formulazione di proposte e progetti per i giovani, con budget e spazi dedicati; 

\fakeparagraph{Ripristinare il Consiglio Comunale dei Ragazzi e delle Ragazze}; 

\fakeparagraph{Prevenire il disagio giovanile in tutte le sue forme}
\begin{itemize}
  \item mappando, grazie ad un confronto costante tra i Servizi sociali, le scuole, gli oratori, gli enti e le associazioni che operano sul territorio, le criticità, soprattutto dei giovani privi di una rete familiare solida;
  \item costruendo una progettualità dedicata che preveda attività, momenti aggregativi ed eventi, ma anche spazi dedicati a giovani di ogni frazione e provenienza sociale;
\end{itemize}

\fakeparagraph{Supportare l'indipendenza abitativa} attraverso incentivi al pagamento di affitti e mutui strutturati ad hoc, ragionando anche sulle giovani coppie per stimolare un progetto sovracomunale a contrasto del calo demografico;

\fakeparagraph{Creare un centro giovanile}, dedicando a questo scopo uno spazio comunale parzialmente autogestito (e.g. in collaborazione con realtà nuove o già esistenti come GeT), in cui organizzare corsi, attività, mostre ed eventi in collaborazione con le realtà già attive sul territorio;

\fakeparagraph{Far ripartire il progetto Dalmine Young}, ipotizzando un budget per quest'ultimo e conferendogli obiettivi chiari e precisi, in modo che nel tempo possa affermarsi ed espandersi come contenitore dell'offerta formativa, culturale, lavorativa e di tempo libero per i giovani;

\fakeparagraph{Supportare i lavori del nuovo spazio Informagiovani} affinché riesca a coinvolgere l'ambito territoriale nella proposizione ed incentivazione di politiche giovanili.
