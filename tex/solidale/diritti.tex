% !TeX root = ../../main.tex
\section{Diritti e Inclusione}
\fakeparagraph[]{Contrastare ogni forma di discriminazione incluse quelle sessuali e di genere}
\begin{itemize}
  \item con campagne di sensibilizzazione dedicate; 
  \item attraverso l'adesione del Comune di Dalmine al Bergamo Pride;
  \item con la presenza di un professionista dedicato ai membri della comunità LGBTQIA+ nei centri antiviolenza;
  \item formando il personale comunale e scolastico su tematiche di orientamento sessuale ed identità di genere;
  \item attuando pienamente la legislazione in materia di unioni civili e riconoscimento dei figli delle coppie omogenitoriali; 
\end{itemize}

\begin{bluebox}
\fakeparagraph[]{Contrastare la violenza sulle donne}
\begin{itemize}
  \item finanziando maggiormente i centri già esistenti;
  \item attraverso l'apertura di uno sportello comunale che interagisca con Associazioni e centri presenti sul territorio, con la Rete Interistituzionale Antiviolenza e con le ASST;
  \item creando nuovi progetti di housing sicuro per vittime di violenza;
  \item realizzando percorsi educativi, culturali e informativi nelle scuole e con le famiglie su temi come l'educazione al consenso;
  \item aumentando la formazione dei dipendenti comunali, soprattutto tra quelli operanti nei Servizi Sociali e nella Polizia Locale;
\end{itemize}
\end{bluebox}

\fakeparagraph[]{Realizzare campagne informative efficaci sul testamento biologico} e sulla possibilità di depositarlo presso il Comune.

\fakeparagraph{Migliorare l'integrazione della popolazione straniera}
\begin{itemize}
  \item rendendo disponibile il sito web comunale anche in altre lingue;
  \item istituendo servizi di supporto alla vita quotidiana come uno sportello comunale sull'immigrazione;
  \item attraverso corsi di italiano ed economia domestica, assistenza digitale e burocratica;
  \item contrastando l'esclusione per mezzo di progetti ed iniziative di scambio culturale istituendo ad esempio un Festival delle Culture che valorizzi tutte quelle presenti sul territorio;
\end{itemize}
