% !TeX root = ../../main.tex
\section{Associazionismo}
Come dimostrato dalla recente pandemia, la rete di associazioni e volontari che dà vita al Terzo Settore è un patrimonio inestimabile che va tutelato, sostenuto e promosso, perché in grado di raggiungere in modo capillare numerose persone e soddisfare diverse necessità. Tuttavia sono anni che si discute della crisi che investe questo ambito senza che la politica si sia mossa con sufficiente convinzione per incanalare lo spirito solidale della comunità dalminese. Per farlo ci proponiamo di:

\fakeparagraph{Avvicinare il Comune alle Associazioni}, attraverso l'istituzione di una delega all'associazionismo, che possa essere conferita ad un assessore o a un consigliere, e la creazione di un tavolo permanente di co-progettazione e dialogo, che riunisca la Consulta delle Associazioni, tutte le Associazioni del territorio, ma anche i gruppi informali di volontari attivi a Dalmine;

\fakeparagraph{Dare supporto operativo} ma non invasivo alle associazioni e ai gruppi informali, attraverso una collaborazione fattiva con il Centro Servizi per il Volontariato di  Bergamo ed il coinvolgimento di Servizi Sociali, Ufficio Cultura e Terzo Settore, prevedendo anche figure professionali esterne che possano essere referenti. Vanno forniti alle associazioni: spazi dedicati, assistenza in termini di comunicazione, organizzazione, digitalizzazione e burocrazia, supporto nella realizzazione di eventi di promozione e reclutamento sul territorio. Inoltre intendiamo incrementare i fondi ad esse dedicati;

\fakeparagraph{Creare un portale comunale, fisico e virtuale, delle Associazioni}, che faccia da guida per i cittadini, in modo che i contatti, i servizi erogati ed i passaggi burocratici necessari siano a disposizione di tutti e le fasce in difficoltà possano essere supportate passo per passo e godere di servizi efficaci;

\fakeparagraph{Rendere più capillare la comunicazione} relativa alle Associazioni ed al Registro dei Volontari, pensando ad esempio a momenti di confronto sul tema in tutte le frazioni e creando uno spazio dedicato su Informa Dalmine;

\fakeparagraph{Assicurare la gratificazione dei volontari} affinché il volontariato non venga solamente visto come sacrificio, ipotizzando ad esempio una revisione del sistema delle benemerenze ed eventi sul tema;

\fakeparagraph{Lavorare con le scuole} affinché, fin dall'infanzia, si sottolinei l'importanza del coinvolgimento nella vita della propria comunità e della solidarietà. In particolare va stimolata, con progetti dedicati, la partecipazione degli studenti delle scuole superiori alla vita culturale e associazionistica di Dalmine, andando di fatto a creare un potenziale bacino di nuovi protagonisti del Terzo Settore.
