% !TeX root = ../../main.tex
\section{Salute}
Vogliamo una Dalmine che abbia al centro le persone ed il loro benessere. Riteniamo dunque necessario:

\begin{bluebox}
\fakeparagraph[]{Affrontare il problema dei medici e dei pediatri di base}
\begin{itemize}
  \item insistendo perché gli enti competenti (e.g. Regione Lombardia) trovino soluzioni concrete nel breve e nel lungo termine, incrementando il numero e rendendo più appetibili le borse di studio per specializzandi in questo settore;
  \item promuovendo l'insediamento di studi medici nelle frazioni meno servite, calmierando gli affitti ed abbassando l'IMU;
  \item ipotizzando la possibilità di dedicare una struttura pubblica a studi consorziati di medici e pediatri;
\end{itemize}
\end{bluebox}

\begin{bluebox}
\fakeparagraph[]{Porre più attenzione alla salute mentale}
\begin{itemize}
  \item mappando a livello di Ambito Territoriale le problematiche rilevanti;
  \item promuovendo un servizio diurno per persone con disturbi psichiatrici o problematiche psicosociali;
  \item creando un punto di ascolto dedicato alle fragilità della popolazione ma soprattutto a carico dei giovani, come i disturbi del comportamento alimentare e quelli legati all'ansia;
  \item lavorando in accordo con Regione Lombardia affinché nella Casa della Comunità venga inserito uno Psicologo di base, che possa anche integrarsi con le realtà del Terzo settore creando una rete di supporto a chi vive situazioni di fragilità;
  \item potenziando i servizi già attivi nelle scuole di ogni ordine e grado;
  \item istituendo e rafforzando i gruppi di auto mutuo aiuto (AMA) che si occupano di supportare coloro che hanno vissuto traumi di vario genere;
\end{itemize}
\end{bluebox}

\fakeparagraph[]{Promuovere l'educazione alla salute}
\begin{itemize}
  \item dando visibilità alle iniziative promosse dalla Casa di Comunità;
  \item con la diffusione di una corretta informazione scientifica (e.g. campagne mirate sulle vaccinazioni, educazione alimentare), intervenendo anche a livello scolastico; 
  \item anche sul tema della salute riproduttiva e sessuale e dei diritti ad essa correlati, costruendo  progetti di educazione alla sessualità e all'affettività;
  \item attraverso iniziative di prevenzione delle malattie sessualmente trasmissibili (MST);
\end{itemize}

\fakeparagraph{Collaborare a livello sovracomunale} per individuare soluzioni alternative e preventive affinché i servizi socio-assistenziali attualmente portati avanti da gruppi di volontari molto anziani o poco numerosi continuino ad essere garantiti in modo strutturale in futuro;

\begin{bluebox}
\fakeparagraph[]{Continuare nel contrasto alla ludopatia ed alle dipendenze}
\begin{itemize}
  \item con un occhio di riguardo verso gli anziani, ad esempio riducendo gli orari in cui è consentito utilizzare slot machine e videolottery e aumentando gli incentivi per chi le rimuove;
  \item sensibilizzando e rendendo i giovani più consapevoli dei rischi derivanti dal gioco d'azzardo online e dell'abuso di sostanze che creano dipendenza attraverso progetti educativi dedicati;
  \item collaborando strettamente con i Servizi per le Dipendenze Patologiche (SerD) presenti sul territorio provinciale, anche considerando la precocità degli abusi. 
\end{itemize}
\end{bluebox}
