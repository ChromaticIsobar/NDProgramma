% !TeX root = ../../main.tex
\section{Politiche Abitative}
La casa è un tassello fondamentale delle politiche di welfare perché ogni cittadino deve poter vivere in un luogo sicuro ed in alloggi dignitosi.
Vogliamo quindi:

\begin{bluebox}
\fakeparagraph[]{Difendere il diritto alla casa}
\begin{itemize}
  \item attraverso la promozione dei programmi di edilizia pubblica e popolare già attivi ed arricchendo questi ultimi con un nuovo piano per far fronte alle emergenze e con un fondo economico dedicato (e.g. garantendo un primo periodo gratuito, dotandosi di appartamenti comunali disponibili per emergenze temporanee, creando convenzioni con alberghi e privati);
  \pagebreak
  \item istituendo convenzioni ed incentivi sovracomunali per coloro che intendano mettere a disposizione i propri immobili per housing sociale/affitti calmierati;
  \item prevedendo una specifica voce per chi costruisce nuovi edifici ed incentivando chi si trasferisce in locali sfitti da tempo;
  \item facendo in modo che il Comune possa intercedere o agire da garante verso affittuari a basso reddito o con contratti di lavoro a  tempo determinato, che diversamente avrebbero difficoltà a sottoscrivere un contratto;
\end{itemize}
\end{bluebox}

\fakeparagraph[]{Creare un fondo a sostegno degli affitti}
per i giovani, le giovani coppie, i nuclei familiari e le fasce di popolazione più fragili, come aiuto temporaneo per rafforzarne l'autonomia o per evitare di cadere in condizioni di povertà;

\fakeparagraph{Promuovere nuovi progetti di autonomia abitativa e housing sociale};

\fakeparagraph{Incentivare i privati} a mettere a disposizione alloggi sfitti per situazioni di fragilità e per facilitare l'accesso alla casa agli studenti;

\fakeparagraph{Prevedere progetti e strumenti che facilitino l'accesso alla casa}, realizzando per esempio un portale web dedicato;

\fakeparagraph{Riqualificare edifici in degrado per la creazione di residenze per studenti}, rilanciando così frazioni periferiche che stanno vivendo problematiche di svuotamento.
