% !TeX root = ../../main.tex
\section{Politiche per la Terza Età}
Mentre assistiamo ad un calo costante del tasso di natalità, il numero di anziani sta crescendo e con esso anche i relativi bisogni. Accanto alle strategie a vantaggio dell'aumento demografico, è necessario rivedere le politiche per le persone anziane, con interventi dedicati come:

\fakeparagraph{Convertire il Centro Diurno Anziani in un Centro per Tutte le Età} (CTE), trasformandolo in un luogo di aggregazione quotidiana e di scambio intergenerazionale per contrastare il fenomeno della solitudine;

\begin{bluebox}
\fakeparagraph[]{Promuovere l'invecchiamento attivo}, al fine di valorizzare il ruolo delle persone anziane nella comunità e la loro partecipazione alla vita sociale, civica, economica e culturale ad esempio
\begin{itemize}
  \item organizzando eventi culturali dedicati;
  \item promuovendo la partecipazione ad attività di volontariato;
  \item garantendo la formazione e l'aggiornamento dei soggetti che volontariamente operano in favore delle persone anziane; 
  \item sostenendo la persona anziana nel proprio contesto familiare, abitativo e territoriale agevolando una vita di relazione attiva;
  \item promuovendo  azioni tese al mantenimento del benessere durante l'invecchiamento, sostenendo la diffusione di corretti stili di vita così come l'educazione motoria e fisica;
\end{itemize}
\end{bluebox}

\fakeparagraph[]{Stimolare la condivisione di esperienze e la formazione} % in diversi ambiti 
attraverso momenti di incontro e scambio intergenerazionale;

\fakeparagraph{Sperimentare il progetto Condominio Sociale} che, supportato e supervisionato con l'aiuto dai servizi sociali, prevede 
\begin{itemize}
  \item un'attività di sorveglianza solidale in condomini in cui risiedono inquilini anziani soli da parte di vicini volontari;  
  \item visite di cortesia al domicilio degli anziani soli per contrastare l'isolamento sociale, organizzate in collaborazione con le associazioni del territorio e la struttura del CTE;
\end{itemize}

\fakeparagraph{Incrementare il numero e la varietà di servizi offerti agli anziani}, ad esempio potenziando le convenzioni con le strutture di ricovero ed il servizio di trasporto dedicato e implementando progetti comunali di assistenza digitale, burocratica e di contrasto alle truffe, oltre all'introduzione graduale di una navetta per gli spostamenti quotidiani.
