% !TeX root = ../../main.tex
\section{Cultura}
Vogliamo una Dalmine che non sia una città dormitorio, in cui la cultura rappresenti un bene comune, prosperi e sia pienamente inclusiva, ovvero accessibile a tutti e l'espressione di tutti. Serve quindi attuare una serie di azioni che valorizzino l'esistente e stimolino la creazione del nuovo, promuovendo una rassegna di qualità che veda lo scambio tra culture e generazioni, coinvolga attivamente la popolazione e offra ad artisti e associazioni sostegno nella realizzazione delle proprie iniziative. Vogliamo perciò:

\fakeparagraph{Investire nella riqualificazione di spazi} associativi, culturali e aggregativi al chiuso ed all'aperto, affidandoli poi alle realtà virtuose del territorio per arricchire il tessuto culturale e sociale di Dalmine. Esempi sono l'ex Biblioteca di via Kennedy e quella dei ragazzi di Sforzatica S. Andrea, la Torre Camozzi, l'attuale sede della Polizia Locale (per cui è già previsto il trasferimento presso l'edificio ex Cral) e l'area feste di via Stella Alpina. In particolare presso quest'ultima immaginiamo uno spazio dedicato a rassegne e festival e/o uno spazio estivo eventualmente gestito da privati. Vogliamo inoltre investire risorse nell'ammodernamento del Teatro Civico e della sala prove;

\fakeparagraph{Aumentare il numero delle sale civiche}, in modo tale che siano presenti in ogni frazione, e  ripensare la loro destinazione d'uso come luoghi di aggregazione quotidiana. Ci proponiamo anche di riqualificare le sale civiche già presenti sul territorio; 

\begin{bluebox}
\fakeparagraph[]{Promuovere percorsi, iniziative, festival e eventi su diverse tematiche culturali e di interesse sociale}, come quelle legate alla salute, alla scienza, all'ambiente ed alla tecnologia. Pensiamo a:
\begin{itemize}
  \item una rassegna più vivace e partecipata che coinvolga tutte le fasce d'età;
  \item eventi ed occasioni culturali e di incontro su tutto il territorio, privilegiando non solo il centro di Dalmine ma anche le frazioni;
  \item una ``Giornata dell'Arte'' dedicata agli artisti dalminesi, che sia allo stesso tempo una vetrina e un'occasione di ricevere finanziamenti per la propria attività; 
\end{itemize}
\end{bluebox}

\fakeparagraph[]{Pianificare interventi di efficientemente energetico, termico e sismico della biblioteca "Rita Levi Montalcini"}, con rimozione delle barriere architettoniche. Ci impegneremo per consolidare il ruolo della biblioteca come cuore pulsante delle attività culturali, un luogo capace di accogliere tutte le fasce di età e di offrire un palinsesto culturale variegato. Intendiamo incrementare gli orari di apertura per andare incontro alle esigenze dei diversi fruitori, in particolare nell'ottica dell'aumento del numero di spazi studio sul territorio;

\fakeparagraph{Creare rete con il Comune di Bergamo e la Provincia} in modo tale che la rassegna culturale possa trarre benefici reciproci e che Dalmine possa essere percepita meno come periferia. Nella stessa ottica proponiamo anche il gemellaggio con una città Europea.

\fakeparagraph{Realizzare uno spazio verde della cultura}, riqualificando uno dei parchi cittadini più grandi ed installando tavoli, chioschi, scacchiere ed attrezzature sportive;

\fakeparagraph{Valorizzare la storia e quindi l'offerta turistica della nostra città}, in collaborazione con enti ed Associazioni e diversificando la promozione del territorio;
