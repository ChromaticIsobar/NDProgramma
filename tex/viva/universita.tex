% !TeX root = ../../main.tex
\section{Università}
Negli ultimi mesi si è paventata la possibilità di uno spostamento da Dalmine della Scuola di Ingegneria dell'Università degli Studi di Bergamo. È importante che l'Amministrazione convogli gli sforzi affinché questo trasferimento non avvenga, dato che la presenza dell'Università impatta su svariati aspetti della vita del nostro comune: la mobilità pubblica e privata, il commercio, la ristorazione, il mercato immobiliare ed i servizi. Vogliamo perciò valorizzare la presenza dell'Università sul nostro territorio, creando rapporti stabili e continuativi con quest'ultima, coinvolgendo non solo il Magnifico Rettore e gli organi decisionali ma anche gli studenti. Intendiamo:

\fakeparagraph{Dedicare un assessorato alla Città dell'Istruzione}, così da avere una figura che, oltre a ricoprire un ruolo chiave nell'istruzione a tutti i livelli, si occupi in modo continuativo di interagire con l'Università degli Studi di Bergamo. Questo per generare una vera collaborazione tra l'Università e l'ente pubblico anche al fine di costruire una visione di governo innovativa;

\fakeparagraph{Stipulare un Patto dell'Istruzione}, un accordo di pianificazione congiunta che coinvolga tutte le scuole presenti sul territorio, da quella primaria fino all'Università. Tale patto dovrebbe regolare la collaborazione tra gli enti, definire periodicamente strategie, obiettivi ed incarichi reciproci su temi importanti per la città e verificare in modo periodico l'andamento delle azioni pianificate;

\fakeparagraph{Collaborare con l'Università per individuare nuovi spazi utili per la didattica}, partendo da un'attenta progettazione degli spazi pubblici che consenta in ultima analisi di ridisegnare il campus. A tal fine sarà anche necessario tenere sotto osservazione il processo di riqualificazione dell'Ex Centrale Enel di viale Marconi;

\fakeparagraph{Adeguare la viabilità}, fornendo agli studenti valide alternative all'automobile per raggiungere i luoghi cardine dell'Università. Intendiamo infatti stimolare gli universitari ad utilizzare i parcheggi decentrati e a sfruttare i mezzi alternativi che verranno messi a disposizione per raggiungere le sedi universitarie. Vogliamo inoltre porre le basi affinché il trasporto pubblico locale e la mobilità dolce, anche intercomunale, diventino le due primarie modalità di trasporto per gli studenti da e verso Dalmine;

\fakeparagraph{Coinvolgere gli studenti nella vita del nostro Comune}, che deve diventare più attrattivo, con queste modalità:
\begin{itemize}
  \item \fakeparagraph[]{la cooperazione attiva tra Comune e studenti universitari}, prevedendo incontri, tirocini, progetti formativi e di tesi che siano altamente qualificanti per gli studenti e vadano a beneficio del Comune e dei servizi da esso offerti, anche in partnership con attori del territorio;
  \item \fakeparagraph[]{l'aumento del numero di spazi studio} sfruttando edifici del comune e spazi all'aperto (come il parco Camozzi), ma anche realizzando partnership con privati e scuole;
  \item \fakeparagraph[]{la promozione di  una rassegna culturale} che veda gli universitari non solo come fruitori dei servizi proposti, ma anche come parte attiva in ambito culturale, di volontariato e associazionismo, hobbistico, sportivo e politico;
  \item \fakeparagraph[]{il consolidamento del rapporto con il Centro Universitario Sportivo} (CUS), pensando anche ad una collaborazione con le piscine di Dalmine;
  \item \fakeparagraph[]{la creazione di una rete commerciale avente come target gli studenti} con scontistiche e convenzioni dedicate al fine di fidelizzare questi ultimi e accrescere il tessuto economico della città;
  \item \fakeparagraph[]{progetti di baratto universitario}, creando una rete di volontari tra studenti per aiutare a gestire la biblioteca ed altri spazi di pubblica utilità, in cambio di crediti formativi aggiuntivi;
  \item \fakeparagraph[]{la messa a disposizione di strumenti che facilitino l'accesso alla casa per gli studenti} \begin{itemize}
    \item \fakeparagraph[]{creando un portale dedicato} ed incentivando l'affitto a canoni concordati;
    \item \fakeparagraph[]{riqualificando edifici in degrado} per la creazione di residenze per studenti;
    \item \fakeparagraph[]{rilanciando frazioni periferiche} con problematiche di alloggi sfitti;
    \item \fakeparagraph[]{realizzando progetti sociali} che prevedono l'affitto di stanze a prezzi calmierati a fronte di piccoli servizi alle persone anziane proprietarie così anche da contrastare l'isolamento di queste ultime.
  \end{itemize}
\end{itemize}
