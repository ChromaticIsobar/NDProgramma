% !TeX root = ../../main.tex
\section{Scuola}
La scuola rappresenta uno dei principali luoghi di formazione dei futuri cittadini. Per questo è necessario  impegnarsi affinché si intensifichi il dialogo con le istituzioni scolastiche, ponendo al centro i valori di legalità, senso civico, rispetto e solidarietà, che costituiscono la base dell'essere comunità. Perché questo si realizzi intendiamo:

\fakeparagraph{Organizzare, sin dalla scuola primaria, incontri cadenzati negli istituti} in collaborazione con Polizia Locale, associazioni e professionisti, per formare ed informare gli studenti  su tematiche sensibili e di attualità come il bullismo ed il cyberbullismo, la salute mentale, l'educazione all'affettività, alla sessualità, al digitale ed ai social media, alla legalità ed alla cittadinanza attiva;

\fakeparagraph{Migliorare i servizi di orientamento} per gli studenti delle scuole secondarie di secondo grado, anche attraverso eventi innovativi sul territorio che ad esempio prevedano il coinvolgimento di aziende ed associazioni di categoria, oltre che il contatto diretto tra studenti;

\begin{bluebox}
\fakeparagraph[]{Tutelare il diritto allo studio}
\begin{itemize}
  \item contrastando la dispersione scolastica con iniziative dedicate; 
  \item aumentando i destinatari delle borse di merito comunali tramite partnership pubblico-private;
  \item prevedendo sussidi per l'acquisto del materiale scolastico, anche promuovendo  pratiche virtuose e sostenibili, come i mercatini comunali dei libri usati prima dell'inizio dell'anno scolastico;
  \item sostenendo gli studenti con disabilità; 
  \item garantendo ad ogni studente la necessaria dotazione informatica, collaborando con privati ed aziende; 
\end{itemize}
\end{bluebox}

\fakeparagraph[]{Rafforzare la Commissione Istruzione}, che deve riunirsi con cadenza regolare, porsi obiettivi concreti e coinvolgere vari attori del territorio nella progettazione dell'offerta e dei servizi;

\fakeparagraph{Incentivare la mobilità sostenibile di studenti e genitori}, proseguendo e amplificando i contributi alle famiglie per abbonamenti di trasporto pubblico locale, soprattutto nel caso di spostamenti dalle frazioni che non hanno scuole, e riattivando servizi come il Piedibus;

\fakeparagraph{Aumentare il numero di  progetti innovativi portati avanti nelle scuole} che ad esempio si contraddistinguano per sostenibilità  ambientale, tra cui Semaforo Azzurro che favorisce il ricircolo d'aria nelle aule;

\fakeparagraph{Attivare progetti di formazione pomeridiani e serali} che costituiscano attività extracurriculari per gli studenti e non solo;

\fakeparagraph{Aumentare gli investimenti sull'asilo nido comunale L'Anatroccolo}, in modo da favorire l'accessibilità del servizio anche promuovendo le agevolazioni ed i bandi relativi;

\fakeparagraph{Valutare la possibilità di aprire ulteriori asili nido comunali} che rispondano meglio alle esigenze dei neogenitori anche in termini di orari;

\fakeparagraph{Aumentare i servizi pre-scuola, doposcuola ed estivi} destinati ai ragazzi delle scuole primarie e secondarie di tutte le frazioni, valutando l'arricchimento dell'offerta comunale anche sostenendo le realtà dell'associazionismo locale e dei gruppi informali di genitori. Queste iniziative si svilupperebbero non solo in ambito strettamente scolastico, ma includendo progetti sportivi, artistici, culturali ed incentrati sullo sviluppo di competenze trasversali;

\fakeparagraph{Investire sull'edilizia scolastica} con un piano di manutenzione e ammodernamento, accompagnato da controlli sulla sicurezza delle strutture e sugli adempimenti in materia di obblighi prevenzionali (e.g. prevenzione incendi, impianti elettrici, impianti termici);

\fakeparagraph{Proseguire il progetto di alfabetizzazione ed istruzione per adulti}, realizzato dal Centro Provinciale Istruzione Adulti (CPIA) di Treviglio in collaborazione con l'Associazione Il Porto. 
