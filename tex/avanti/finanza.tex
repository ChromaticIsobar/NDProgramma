% !TeX root = ../../main.tex
\section{Finanza}
Trasparenza ed onestà sono caratteristiche indispensabili per una corretta Amministrazione finanziaria del Comune. Il coinvolgimento della collettività, la gestione sana e prudente delle entrate, la programmazione adeguata delle uscite e la rimodulazione di tributi, tariffe ed addizionali comunali sono i principi su cui baseremo la nostra gestione. Intendiamo impegnarci per:

\fakeparagraph{Mantenere l'equilibrio tra entrate ed uscite correnti}, utilizzare le entrate straordinarie per investimenti sostenibili, contenere la spesa corrente, evitare residui attivi di difficile realizzabilità;

\fakeparagraph{Limitare la pressione tributaria} senza ridurre i servizi al cittadino;

\begin{bluebox}
\fakeparagraph[]{Eliminare il sistema \textit{flat tax} dell'addizionale comunale IRPEF}, un criterio socialmente ingiusto, rendendola progressiva ed innalzando ulteriormente la soglia di esenzione;
\end{bluebox}

\fakeparagraph[]{Intensificare la lotta all'evasione fiscale} potenziando il controllo incrociato dei dati e collaborando con l'Agenzia delle Entrate, allo stesso tempo recuperando eventuali crediti;

\fakeparagraph{Istituire un Ufficio Bandi}, potenzialmente a livello di Ambito Territoriale, che si occupi della gestione di bandi regionali, nazionali e sovranazionali, comunicando adeguatamente  con le realtà interessate;

\fakeparagraph[\pagebreak]{Contenere le tariffe dei servizi} educativi, sportivi, culturali, sociali e sociosanitari rendendole legate ad indicatori economici ben stabiliti (e.g. ISEE);

\fakeparagraph{Proseguire la partecipazione nelle società Tecnodal, TBSO, ATB, Uniacque e valutare la rinuncia alle quote in Autostrade Bergamasche S.p.A}.
